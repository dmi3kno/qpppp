
%%% OPRE uses endnotes. If you do not use them, put a percent sign before
%%% the \theendnotes command. This template does show how to use them.
%\usepackage{endnotes}
%\let\footnote=\endnote
%\let\enotesize=\normalsize
%\def\notesname{Endnotes}%
%\def\makeenmark{$^{\theenmark}$}
%\def\enoteformat{\rightskip0pt\leftskip0pt\parindent=1.75em
%	\leavevmode\llap{\theenmark.\enskip}}

% Private macros here (check that there is no clash with the style)
% figure packages
%\usepackage{graphicx} % already handled in graphics.tex
\usepackage{eqndefns-left}
%\usepackage{multirow} % already handled in tables.tex
\usepackage{hhline}

% caption package
\usepackage[small, margin=1cm]{caption}

% appendix package
\usepackage{appendix}
% color packages
\usepackage{color}
\definecolor{strcolor}{rgb}{0.6, 0.2, 0.6}
\definecolor{commentcolor}{rgb}{0.3125, 0.5, 0.3125}
\definecolor{keycol}{rgb}{0, 0, 1}


% revision
\newcommand{\rev}[1]{{\color{red} #1}}

% math packages
%\usepackage{amssymb}
%\usepackage{amsmath}
\usepackage{bbm}

% Code package
\usepackage{listings}
\lstset{
	emph={ROVar, ROUn, ROVarDR, ROExpr, RONormInf, RONorm1, RONorm2,ROConstraint,ROExpect, ROSq, ROConstraintSet,ROIntVar,ROBinVar, ROInfinity,ROModel,ROVarDRArray, ROVarArray, ROMinimize,ROUnArray, ROAbs, ROPos, ROSum, int},
	emphstyle={\color{strcolor}\bfseries},
	keywordstyle={\color{blue}\bfseries},
	commentstyle={\color{commentcolor}},
	stringstyle={\color{strcolor}\bfseries},
	language=C++,                % choose the language of the code
	basicstyle={\ttfamily\footnotesize}, % the size of the fonts that are used for the code
	numbers=left,                   % where to put the line-numbers
	numberstyle=\footnotesize,      % the size of the fonts that are used for the line-numbers
	stepnumber=1,                   % the step between two line-numbers. If it's 1 each line will be numbered
	numbersep=5pt,                  % how far the line-numbers are from the code
	backgroundcolor=\color{white},  % choose the background color. You must add \usepackage{color}
	showspaces=false,               % show spaces adding particular underscores
	showstringspaces=false,         % underline spaces within strings
	showtabs=false,                 % show tabs within strings adding particular underscores
	frame=single,	                	% adds a frame around the code
	tabsize=2,	                		% sets default tabsize to 2 spaces
	captionpos=b,                   % sets the caption-position to bottom
	breaklines=true,                % sets automatic line breaking
	breakatwhitespace=false,        % sets if automatic breaks should only happen at whitespace
	escapeinside={\%*}{*)},         % if you want to add a comment within your code
	keywords=[1]{for, break, if, else, function}
}
\renewcommand{\lstlistingname}{Code Segment}

% hyperlinks packages
%\usepackage{hyperref}
\usepackage{url}

% numbering
%\numberwithin{equation}{section}
%\numberwithin{table}{section}
%\numberwithin{figure}{section}

% Equation environments
\newcommand {\bea}{\begin{eqnarray}}
	\newcommand {\eea}{\end{eqnarray}}
\newcommand {\E}[1]{\mathrm{E}\left( #1 \right)}
\newcommand {\Ep}[2]{{\mathrm{E}_{\mathbb{P}_{#1}} \left( #2 \right)}}
\newcommand {\supEp}[1]{\displaystyle \sup_{\mathbb{P} \in \mathbb{F}} \Ep{}{#1}}
\newcommand {\supEpf}[2]{\displaystyle \sup_{\mathbb{P} \in \mathbb{F}_{#1}} \Ep{}{#2}}
\newcommand {\pos}[1]{\paren{#1}^+}
\renewcommand {\neg}[1]{\paren{#1}^-}
\newcommand {\pibound}[1]{\ensuremath{\pi^{#1}\paren{r^0, \mb{r}}}}
\newcommand {\etabound}[1]{\ensuremath{\eta^{#1}\paren{r^0, \mb{r}}}}
\newcommand \conv {\mathrm{conv}}
\newcommand {\p}{{\rm P}}
% mb
\newcommand{\mb}[1]{\mbox{\boldmath \ensuremath{#1}}}
\newcommand{\mbt}[1]{\mb{\tilde{#1}}}
\newcommand{\mbb}[1]{\mb{\bar{#1}}}
\newcommand{\mbbs}[1]{\mbb{\scriptstyle{#1}}}
\newcommand{\mbh}[1]{\mb{\hat{#1}}}
\newcommand{\mbth}[1]{\mbt{\hat{#1}}}
\newcommand{\mbc}[1]{\mb{\check{#1}}}
\newcommand{\mbtc}[1]{\mbt{\check{#1}}}
\newcommand{\mbs}[1]{\mb{\scriptstyle{#1}}}
\newcommand{\mbst}[1]{\mbs{\tilde{#1}}}
\newcommand{\mbsh}[1]{\mbs{\hat{#1}}}
% mc
%\newcommand{\mc}[1]{\mbox{\ensuremath{\mathcal{#1}}}}
\newcommand{\mch}[1]{\hat{\mc{#1}}}
\newcommand{\mcs}[1]{\mc{\scriptstyle{#1}}}
\newcommand{\mcss}[1]{\mc{\scriptscriptstyle{#1}}}
\newcommand{\mcsh}[1]{\hat{\mcs{#1}}}
\newcommand{\mbss}[1]{{\mbox{\boldmath \tiny{$#1$}}}}
\newcommand{\eucnorm}[1]{\left\| #1 \right\|_2}
\newcommand{\dpv}{\displaystyle \vspace{3pt}}
\newcommand{\diag}[1]{\textbf{diag}\paren{#1}}
\newcommand{\yldrk}{\mb{y}^k\paren{\mbt{z}}}
\newcommand{\abs}[1]{\left| #1 \right|}
\DeclareMathOperator{\CVaR}{CVaR}
\DeclareMathOperator{\VaR}{VaR}
%\DeclareMathOperator{\argmin}{\arg\min}
% combinations
\renewcommand{\mbc}[1]{\mb{\mc{#1}}}
%misc
\newcommand{\ceil}[1]{\left\lceil #1  \right\rceil}

%\newtheorem{theorem}{Theorem}
%\newtheorem{lemma}[theorem]{Lemma}
\newtheorem{conj}{Conjecture}
\newtheorem{coro}{Corollary}
%\newtheorem{claim}{Claim}
\newtheorem{Defi}{Definition}
\newtheorem{algorithm}{Algorithm}
%\newtheorem{assumption}{Assumption}

\newcommand{\eg}{\textit{e.g.}}
\newcommand{\ie}{\textit{i.e.}}
\renewcommand{\Re}{\mathbb{R}}

%\renewcommand{\bigtimes}{\mathop{\rm \text{\Large{$\times$}}}}

\def\blot{\quad \mbox{$\vcenter{ \vbox{ \hrule height.4pt
				\hbox{\vrule width.4pt height.9ex \kern.9ex \vrule width.4pt}
				\hrule height.4pt}}$}}

% Natbib setup for author-year style
\usepackage{natbib}
\bibpunct[, ]{(}{)}{,}{a}{}{,}%
\def\bibfont{\fontsize{8}{9.5}\selectfont}%
\def\bibsep{0pt}%
\def\bibhang{16pt}%
\def\newblock{\ }%
\def\BIBand{and}%

%% Setup of theorem styles. Outcomment only one.
%% Preferred default is the first option.
\TheoremsNumberedThrough     % Preferred (Theorem 1, Lemma 1, Theorem 2)
%\TheoremsNumberedByChapter  % (Theorem 1.1, Lema 1.1, Theorem 1.2)
\ECRepeatTheorems

%% Setup of the equation numbering system. Outcomment only one.
%% Preferred default is the first option.
\EquationsNumberedThrough    % Default: (1), (2), ...
%\EquationsNumberedBySection % (1.1), (1.2), ...

% In the reviewing and copyediting stage enter the manuscript number.
%\MANUSCRIPTNO{} % When the article is logged in and DOI assigned to it,
%   this manuscript number is no longer necessary

%\newdimen\setvrulersecondcolumnheightdimen%
%\newbox\setvrulersecondcolumnheightdimenbox%
%%%%%%%%%%%%%%%%%
\gdef\AQ#1{}
\gdef\CQ#1{}
