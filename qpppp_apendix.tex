% Options for packages loaded elsewhere
\PassOptionsToPackage{unicode}{hyperref}
\PassOptionsToPackage{hyphens}{url}
\PassOptionsToPackage{dvipsnames,svgnames,x11names}{xcolor}
%
\documentclass[
  letterpaper,
  DIV=11,
  numbers=noendperiod]{scrartcl}

\usepackage{amsmath,amssymb}
\usepackage{iftex}
\ifPDFTeX
  \usepackage[T1]{fontenc}
  \usepackage[utf8]{inputenc}
  \usepackage{textcomp} % provide euro and other symbols
\else % if luatex or xetex
  \usepackage{unicode-math}
  \defaultfontfeatures{Scale=MatchLowercase}
  \defaultfontfeatures[\rmfamily]{Ligatures=TeX,Scale=1}
\fi
\usepackage{lmodern}
\ifPDFTeX\else  
    % xetex/luatex font selection
\fi
% Use upquote if available, for straight quotes in verbatim environments
\IfFileExists{upquote.sty}{\usepackage{upquote}}{}
\IfFileExists{microtype.sty}{% use microtype if available
  \usepackage[]{microtype}
  \UseMicrotypeSet[protrusion]{basicmath} % disable protrusion for tt fonts
}{}
\makeatletter
\@ifundefined{KOMAClassName}{% if non-KOMA class
  \IfFileExists{parskip.sty}{%
    \usepackage{parskip}
  }{% else
    \setlength{\parindent}{0pt}
    \setlength{\parskip}{6pt plus 2pt minus 1pt}}
}{% if KOMA class
  \KOMAoptions{parskip=half}}
\makeatother
\usepackage{xcolor}
\setlength{\emergencystretch}{3em} % prevent overfull lines
\setcounter{secnumdepth}{-\maxdimen} % remove section numbering
% Make \paragraph and \subparagraph free-standing
\makeatletter
\ifx\paragraph\undefined\else
  \let\oldparagraph\paragraph
  \renewcommand{\paragraph}{
    \@ifstar
      \xxxParagraphStar
      \xxxParagraphNoStar
  }
  \newcommand{\xxxParagraphStar}[1]{\oldparagraph*{#1}\mbox{}}
  \newcommand{\xxxParagraphNoStar}[1]{\oldparagraph{#1}\mbox{}}
\fi
\ifx\subparagraph\undefined\else
  \let\oldsubparagraph\subparagraph
  \renewcommand{\subparagraph}{
    \@ifstar
      \xxxSubParagraphStar
      \xxxSubParagraphNoStar
  }
  \newcommand{\xxxSubParagraphStar}[1]{\oldsubparagraph*{#1}\mbox{}}
  \newcommand{\xxxSubParagraphNoStar}[1]{\oldsubparagraph{#1}\mbox{}}
\fi
\makeatother


\providecommand{\tightlist}{%
  \setlength{\itemsep}{0pt}\setlength{\parskip}{0pt}}\usepackage{longtable,booktabs,array}
\usepackage{calc} % for calculating minipage widths
% Correct order of tables after \paragraph or \subparagraph
\usepackage{etoolbox}
\makeatletter
\patchcmd\longtable{\par}{\if@noskipsec\mbox{}\fi\par}{}{}
\makeatother
% Allow footnotes in longtable head/foot
\IfFileExists{footnotehyper.sty}{\usepackage{footnotehyper}}{\usepackage{footnote}}
\makesavenoteenv{longtable}
\usepackage{graphicx}
\makeatletter
\def\maxwidth{\ifdim\Gin@nat@width>\linewidth\linewidth\else\Gin@nat@width\fi}
\def\maxheight{\ifdim\Gin@nat@height>\textheight\textheight\else\Gin@nat@height\fi}
\makeatother
% Scale images if necessary, so that they will not overflow the page
% margins by default, and it is still possible to overwrite the defaults
% using explicit options in \includegraphics[width, height, ...]{}
\setkeys{Gin}{width=\maxwidth,height=\maxheight,keepaspectratio}
% Set default figure placement to htbp
\makeatletter
\def\fps@figure{htbp}
\makeatother
% definitions for citeproc citations
\NewDocumentCommand\citeproctext{}{}
\NewDocumentCommand\citeproc{mm}{%
  \begingroup\def\citeproctext{#2}\cite{#1}\endgroup}
\makeatletter
 % allow citations to break across lines
 \let\@cite@ofmt\@firstofone
 % avoid brackets around text for \cite:
 \def\@biblabel#1{}
 \def\@cite#1#2{{#1\if@tempswa , #2\fi}}
\makeatother
\newlength{\cslhangindent}
\setlength{\cslhangindent}{1.5em}
\newlength{\csllabelwidth}
\setlength{\csllabelwidth}{3em}
\newenvironment{CSLReferences}[2] % #1 hanging-indent, #2 entry-spacing
 {\begin{list}{}{%
  \setlength{\itemindent}{0pt}
  \setlength{\leftmargin}{0pt}
  \setlength{\parsep}{0pt}
  % turn on hanging indent if param 1 is 1
  \ifodd #1
   \setlength{\leftmargin}{\cslhangindent}
   \setlength{\itemindent}{-1\cslhangindent}
  \fi
  % set entry spacing
  \setlength{\itemsep}{#2\baselineskip}}}
 {\end{list}}
\usepackage{calc}
\newcommand{\CSLBlock}[1]{\hfill\break\parbox[t]{\linewidth}{\strut\ignorespaces#1\strut}}
\newcommand{\CSLLeftMargin}[1]{\parbox[t]{\csllabelwidth}{\strut#1\strut}}
\newcommand{\CSLRightInline}[1]{\parbox[t]{\linewidth - \csllabelwidth}{\strut#1\strut}}
\newcommand{\CSLIndent}[1]{\hspace{\cslhangindent}#1}

\KOMAoption{captions}{tableheading}
\makeatletter
\@ifpackageloaded{caption}{}{\usepackage{caption}}
\AtBeginDocument{%
\ifdefined\contentsname
  \renewcommand*\contentsname{Table of contents}
\else
  \newcommand\contentsname{Table of contents}
\fi
\ifdefined\listfigurename
  \renewcommand*\listfigurename{List of Figures}
\else
  \newcommand\listfigurename{List of Figures}
\fi
\ifdefined\listtablename
  \renewcommand*\listtablename{List of Tables}
\else
  \newcommand\listtablename{List of Tables}
\fi
\ifdefined\figurename
  \renewcommand*\figurename{Figure}
\else
  \newcommand\figurename{Figure}
\fi
\ifdefined\tablename
  \renewcommand*\tablename{Table}
\else
  \newcommand\tablename{Table}
\fi
}
\@ifpackageloaded{float}{}{\usepackage{float}}
\floatstyle{ruled}
\@ifundefined{c@chapter}{\newfloat{codelisting}{h}{lop}}{\newfloat{codelisting}{h}{lop}[chapter]}
\floatname{codelisting}{Listing}
\newcommand*\listoflistings{\listof{codelisting}{List of Listings}}
\makeatother
\makeatletter
\makeatother
\makeatletter
\@ifpackageloaded{caption}{}{\usepackage{caption}}
\@ifpackageloaded{subcaption}{}{\usepackage{subcaption}}
\makeatother
\ifLuaTeX
  \usepackage{selnolig}  % disable illegal ligatures
\fi
\usepackage{bookmark}

\IfFileExists{xurl.sty}{\usepackage{xurl}}{} % add URL line breaks if available
\urlstyle{same} % disable monospaced font for URLs
\hypersetup{
  pdftitle={Quantile-parameterized distributions for expert knowledge elicitation},
  pdfauthor={Dmytro Perepolkin; Erik Lindström; Ullrika Sahlin},
  colorlinks=true,
  linkcolor={blue},
  filecolor={Maroon},
  citecolor={Blue},
  urlcolor={Blue},
  pdfcreator={LaTeX via pandoc}}

\title{Quantile-parameterized distributions for expert knowledge
elicitation}
\usepackage{etoolbox}
\makeatletter
\providecommand{\subtitle}[1]{% add subtitle to \maketitle
  \apptocmd{\@title}{\par {\large #1 \par}}{}{}
}
\makeatother
\subtitle{Supplementary Materials}
\author{Dmytro Perepolkin \and Erik Lindström \and Ullrika Sahlin}
\date{}

\begin{document}
\maketitle

\section*{Appendix A. Distribution
functions}\label{appendix-a.-distribution-functions}
\addcontentsline{toc}{section}{Appendix A. Distribution functions}

\subsection*{Myerson Distribution}\label{myerson-distribution}
\addcontentsline{toc}{subsection}{Myerson Distribution}

The derivative of the quantile function with respect to the depth \(u\)
is the Quantile Density Function, which for Myerson distribution has the
following form

\[
q(u\vert q_1,q_2,q_3,\alpha)=\begin{cases}
\rho\frac{\beta^\kappa\ln(\beta)}{(\beta-1)}\frac{q_{norm}(u)}{\Phi^{-1}(1-\alpha)}, \; \beta \neq 1\\
\rho\frac{q_{norm}(u)}{\Phi^{-1}(1-\alpha)}, \; \beta = 1
\end{cases}
\]

where \(q_{norm}=\frac{d\Phi^{-1}(u)}{du}\) is the quantile density
function for the standard normal distribution.

The Myerson distribution is invertible. The distribution function of
random variable \(X\) has the form

\[
\begin{gathered}
\psi=\Phi^{-1}(1-\alpha)\left(\frac{\ln\left(1+\frac{(x-q_2)(\beta-1)}{\rho}\right)}{\ln(\beta)}\right)\\
F(x\vert q_1, q_2, q_3, \alpha)=\\
\begin{cases}
\Phi(\psi), \; \beta\neq 1\\
F_{norm}(x\vert q_2,\rho/\Phi^{-1}(1-\alpha) ), \; \beta=1
\end{cases}
\end{gathered}
\]

where \(\Phi()\) is the CDF of the standard normal distribution and
\(\Phi^{-1}()\) is its inverse.
\(F_{norm}(x\vert q_2,\rho/\Phi^{-1}(1-\alpha))\) is the CDF of the
normal distribution with mean \(\mu=q_2\) and standard deviation
\(\sigma=\rho/\Phi^{-1}(1-\alpha)\).

The derivative of the distribution function with respect to the random
variable \(X\) is the probability density function, which for the
Myerson distribution takes the following form

\[
\begin{gathered}
f(x\vert q_1, q_2, q_3, \alpha)=\\
\begin{cases}
\frac{\Phi^{-1}(1-\alpha)(\beta-1)}{(\rho+(x-q_2)(\beta-1))\ln(\beta)}\varphi(\psi), \; \beta\neq1\\
f_{normal}(x\vert q_2,\rho/\Phi^{-1}(1-\alpha)),\; \beta=1
\end{cases}
\end{gathered}
\]

where \(\varphi()\) is the probability density function of the standard
normal distribution,
\(f_{normal}(x\vert q_2,\frac{\rho}{\Phi^{-1}(1-\alpha)})\) is the PDF
of the normal distribution with the mean \(\mu=q_2\) and standard
deviation \(\sigma=\rho/\Phi^{-1}(1-\alpha))\).

\subsection*{Generalized Myerson
Distributions}\label{generalized-myerson-distributions}
\addcontentsline{toc}{subsection}{Generalized Myerson Distributions}

The Quantile Density Function of Generalized Myerson Distribution for
\(u\neq0, u\neq1\) is

\[
\begin{gathered}
q_M(u\vert q_1,q_2,q_3,\alpha)=\\
\begin{cases}
\rho\frac{\beta^\kappa\ln(\beta)}{(\beta-1)}\frac{s(u)}{S(1-\alpha)}, \; \beta \neq 1\\
\rho\frac{s(u)}{S(1-\alpha)}, \; \beta = 1
\end{cases}
\end{gathered}
\]

where \(S(u)\) is the quantile function and \(s(u)=\frac{dS(u)}{du}\) is
the quantile density function for the kernel distribution. When \(u=0\)
or \(u=1\) the \(q_M(u)=\infty\).

The Generalized Myerson distribution is invertible. The distribution
function of random variable \(X\) has the form

\[
\begin{gathered}
\psi =S(1-\alpha)\left(\frac{\ln\left(1+\frac{(x-q_2)(\beta-1)}{\rho}\right)}{\ln(\beta)}\right)\\
F_M(x\vert q_1, q_2, q_3, \alpha) =\\
\begin{cases}
F(\psi), \; \beta\neq 1\\
q_2+ \frac{\rho}{S(1-\alpha)}F(x), \; \beta=1
\end{cases}
\end{gathered}
\]

where \(F()\) is the standard CDF of the kenel distribution and \(S()\)
is its inverse.

The derivative of the distribution function with respect to the random
variable \(X\) is the probability density function, which for the
Myerson distribution takes the following form

\[
\begin{gathered}
f_M(x\vert q_1, q_2, q_3, \alpha)=\\
\begin{cases}
\frac{S(1-\alpha)(\beta-1)}{(\rho+(x-q_2)(\beta-1))\ln(\beta)}f(\psi), \quad &\beta\neq1\\
f\left(\frac{x-q_2}{\rho/S(1-\alpha)}\right),\quad &\beta=1
\end{cases}
\end{gathered}
\]

where \(f()\) is the probability density function of the standard kernel
distribution. Compare it to the simplicity of the Quantile Density
Function above.

\subsection*{Johnson Quantile-Parameterized
Distribution}\label{johnson-quantile-parameterized-distribution}
\addcontentsline{toc}{subsection}{Johnson Quantile-Parameterized
Distribution}

The JQPD-B quantile density function can be computed as

\[
q_B(p)=\begin{cases}
(u_b-l_b)\varphi(\xi+\lambda\sinh(\delta(z(p)+nc))) \times\\
\quad \times \lambda\cosh(\sigma(z(p)+nc)) \sigma q_{norm}(p), \; n\neq 0\\
(u_b-l_b)\varphi\left(B+\left(\frac{H-L}{2c}\right)z(p)\right)
\times \\
\quad \times \left(\frac{H-L}{2c}\right)q_{norm}(p), \; n=0
\end{cases}
\]

The JQPD-B distribution function

\[
F_B(x)=\begin{cases}
\Phi\left((2c/(H-L))(-B+z\left(\frac{x-l}{u-l}\right))\right), \; n=0 \\
\Phi\left(\frac{1}{\delta}\sinh^{-1}\left(\frac{1}{\lambda}\left(z\left(\frac{x-l}{u-l}\right)-\xi\right)\right)-nc\right), \; n\neq0
\end{cases}
\]

The JQPD-B probability density function (PDF) is

\[
\begin{gathered}
f(x)=\begin{cases}
\frac{2c}{(H-L)(u_b-l_b)}\frac{1}{\varphi\left(z\left(\frac{x-l_b}{u_b-l_b}\right)\right)}\varphi\left(\frac{2c}{H-L}\left(-B+z\left(\frac{x-l_b}{u-l_b}\right)\right)\right), \quad &n=0\\
\frac{1}{\delta}\frac{1}{u_b-l_b}\varphi\left(-nc+\frac{1}{\delta}\sinh^{-1}\left(\frac{1}{\lambda}\left(-\xi+z\left(\frac{x-l_b}{u_b-l_b}\right)\right)\right)\right) \times \\ \indent
\frac{1}{\varphi\left(z\left(\frac{x-l_b}{u_b-l_b}\right)\right)}\frac{1}{\sqrt{\lambda^2+\left(-\xi+z\left(\frac{x-l_b}{u_b-l_b}\right)\right)^2}}, \quad &n\neq 0\\
\end{cases}
\end{gathered}
\]

J-QPD-S quantile density function

\[
q_S(p)=\begin{cases}
\theta\exp\left(\lambda\delta z(p)\right)\lambda\delta q_{norm}(p), \quad &n=0\\
\theta\exp\left(\lambda\sinh^{-1}(\delta z(p))+\sinh^{-1}(nc\delta)\right)\lambda\frac{1}{\sqrt{1+(\delta z(p))^2}}\delta q_{norm}(p), \quad &n\neq0\\
\end{cases}
\]

J-QPD-S distribution function

\[
F_S(x)=\begin{cases}
F_{lnorm}(x-l_b\vert \ln(\theta), \frac{H-B}{c}), \quad &n=0\\
\Phi\left(\frac{1}{\delta}\sinh\left(\sinh^{-1}\left(\frac{1}{\lambda}\ln\frac{x-l_b}{\theta}\right)-\sinh^{-1}(nc\delta)\right)\right), \quad &n\neq0\\
\end{cases}
\]

J-QPD-S probability density function (PDF)

\[
f_S(x)=\begin{cases}
\frac{1}{x\sigma\sqrt{2\pi}}\exp\left(-\frac{(\ln x-ln\xi)^2}{2\frac{(H-B)^2}{c^2}}\right), \quad &n=0\\
\varphi\left(\frac{\sinh(\sinh^{-1}(cn\sigma)-\sinh^{-1}(\frac{1}{\lambda}\ln\frac{x-l_b}{\theta}))}{\delta}\right)\frac{\cosh(\sinh^{-1}(cn\delta)-\sinh^{-1}(\frac{1}{\lambda}\ln\frac{x-l_b}{\theta}))}{(x-l_b)\delta\lambda\sqrt{1+\left(\frac{\ln\frac{x-l_b}{\theta}}{\lambda}\right)^2}}, \quad &n \neq 0
\end{cases}
\]

where \(\mu=\ln\xi\) and \(\sigma=\frac{H-B}{c}\).

\subsection*{Metalog distribution}\label{metalog-distribution}
\addcontentsline{toc}{subsection}{Metalog distribution}

This section recapitulates ideas and formulas provided in (Keelin 2016)
with our own notation and minor reinterpretations.

Metalog distribution is created from the logistic quantile function
\(Q(p)=\mu+s\text{logit}(p)\), where \(\mu\) is the mean, \(s\) is
proportional to the standard deviation such that \(\sigma=s\pi/\sqrt3\),
\(p\) is the probability \(p \in [0,1]\). The metalog quantile function
is built by substitution and series expansion of its parameters \(\mu\)
and \(s\) with the polynomial of the form:

\[
\begin{aligned}\;
&\mu=a_1+a_4(p-0.5)+a_5(p-0.5)^2+a_7(p-0.5)^3+a_9(p-0.5)^4+\dots, \\
& s=a_2+a_3(p-0.5)+a_6(p-0.5)^2+a_8(p-0.5)^3+a_{10}(p-0.5)^4+\dots,
\end{aligned}
\]

where \(a_i, \; i \in (1\dots n)\) are real constants. Given a
size-\(m\) QPT \(\{p, q\}_m\), where \(p=\{p_1\dots p_m\}\) and
\(q=\{q_1\dots q_m\}\) the vector of coefficients \(a=\{a_1\dots a_m\}\)
can be determined through the set of linear equations.

\[
\begin{aligned}\;
&q_1=a_1+a_2\text{logit}(p_1)+a_3(p_1-0.5)\text{logit}(p_1)+a_4(p_1-0.5)+\cdots,\\
&q_2=a_1+a_2\text{logit}(p_2)+a_3(p_2-0.5)\text{logit}(p_2)+a_4(p_2-0.5)+\cdots,\\
&\vdots\\
&q_m=a_1+a_2\text{logit}(p_m)+a_3(p_m-0.5)\text{logit}(p_m)+a_4(p_m-0.5)+\cdots.\\
\end{aligned}
\]

In the matrix form, this system of equations is equivalent to
\(q=\mathbb{P}a\), where \(q\) and \(a\) are column vectors and
\(\mathbb{P}\) is a \(m \times n\) matrix:

\[
\mathbb{P} = \left[\begin{array}{lllll}
1  &\text{logit}(p_1) &(p_1-0.5)\text{logit}(p_1) &(p_1-0.5) &\cdots\\
1  &\text{logit}(p_2) &(p_2-0.5)\text{logit}(p_2) &(p_2-0.5) &\cdots\\
   &                  &\vdots\\
1  &\text{logit}(p_m) &(p_m-0.5)\text{logit}(p_m) &(p_m-0.5) &\cdots
\end{array}\right]
\]

If \(m=n\) and \(\mathbb{P}\) is invertible, then the vector of
coefficients \(a\) of this \emph{properly parameterized} metalog QPD can
be uniquely determined by

\begin{equation}\phantomsection\label{eq-nmetalogAsMatrixeq}{
a=\mathbb{P}^{-1}q
}\end{equation}

If \(m > n\) and \(\mathbb{P}\) has a rank of at least \(n\), then the
vector of coefficients \(a\) of the \emph{approximated} metalog QPD, can
be estimated using

\[
a=[\mathbb{P}^T\mathbb{P}]^{-1}\mathbb{P}^Tq
\]

The matrix to be inverted is always \(n \times n\) regardless of the
size \(m\) of QPT used.

Metalog \emph{quantile function} (QF) with \(n\) terms
\(Q_{M_n}(u\vert a)\) can be expressed as

\begin{equation}\phantomsection\label{eq-metalogQFeq}{
Q_{M_n}(u\vert a)=\begin{cases}
a_1+a_2\text{logit}(u), \text{ for } n=2, \\
a_1+a_2\text{logit}(u)+a_3(u-0.5)\text{logit}(u), \text{ for } n=3, \\
a_1+a_2\text{logit}(u)+a_3(u-0.5)\text{logit}(u)+a_4(u-0.5), \text{ for } n=4, \\
Q_{M_{n-1}} + a_n(u-0.5)^{(n-1)/2}, \text{ for odd } n \geq 5, \\
Q_{M_{n-1}} + a_n(u-0.5)^{n/2-1}\text{logit}(u), \text{ for even } n \geq 6, \\
\end{cases}
}\end{equation}

where \(u \in [0,1]\) is the cumulative probability and \(a\) is the
size-\(n\) parameter vector of real constants \(a=\{a_1\dots a_n\}\).

The metalog \emph{quantile density function} (QDF) can be found by
differentiating the Equation~\ref{eq-metalogQFeq} with respect to \(u\):

\begin{equation}\phantomsection\label{eq-metalogQDFeq}{
\begin{gathered}
q_{M_n}(u\vert a)=\begin{cases}
a_2\mathcal I(u), \text{ for } n=2, \\
a_2\mathcal I(u)+a_3\left((u-0.5)\mathcal I(u)+\text{logit}(u) \right), \text{ for } n=3, \\
a_2\mathcal I(u) + a_3\left((u-0.5)\mathcal I(u)+\text{logit}(u) \right)+ a_4,  \text{ for } n=4, \\
q_{M_{n-1}} + 0.5a_n(n-1)(u-0.5)^{(n-3)/2}, \text{ for odd } n \geq 5, \\
q_{M_{n-1}} + a_n((u-0.5)^{n/2-1}\mathcal I(u)+\\ \indent (0.5n-1)(u-0.5)^{n/2-2}\text{logit}(u)), \text{ for even } n \geq 6, \\
\end{cases}
\end{gathered}
}\end{equation}

where \(\mathcal I(u)=[u(1-u)]^{-1}\). The constants \(a\) are feasible
iif \(q_{M_{n}}(u\vert a)>0, \;\forall u \in [0,1]\).

Metalog \emph{density quantile function} (DQF), referred to as the
``metalog pdf'' in (Keelin 2016) can be obtained by
\(f(Q_{M_n}(u\vert a))=[q_{M_n}(u\vert a)]^{-1}\).

Metalog \emph{cumulative distribution function} (CDF)
\(F_{M_n}(x\vert a)\) does not have an explicit form because
\(Q_{M_n}(u\vert a)\) is not invertible (Keelin 2016). It is, however,
possible to approximate \(\widehat Q^{-1}_{M_n}(x\vert a)\) using
approximation.

Metalog distribution is defined for all \(x \in \mathbb R\) on the real
line. (Keelin 2016) provides semi-bounded \emph{log-metalog}, and the
bounded \emph{logit-metalog} variations of the metalog distribution. As
the names suggest, this is achieved through the variable substitution
with \(z=\ln(x-b_l)\) or \(z=-\ln(b_u-x)\) for the semi-bounded case,
and \(z=\ln((x-b_l)/(b_u-x))\) for the bounded case, where \(z\) is
metalog-distributed and \(b_l, b_u\) are the lower and upper limits,
respectively. Substituting one of the transformations into the QF and
QDF functions above, yields semi-bounded or bounded metalog
distribution. For the exact formulae of the log-metalog and
logit-metalog refer to (Keelin 2016).

\subsection*{CSW GLD}\label{csw-gld}
\addcontentsline{toc}{subsection}{CSW GLD}

Quantile density function for the CSW GLD is provided in (Chalabi,
Scott, and Wuertz 2012)

\[
\begin{gathered}
q(u\vert\tilde\sigma,\chi,\xi)= \frac{\tilde\sigma}{S(0.75\vert\chi,\xi)-S(0.25\vert\chi,\xi)}
s(u\vert\chi,\xi) \\
s(u\vert\chi,\xi)=\frac{d}{du}S(u\vert\chi,\xi)=u^{\alpha+\beta-1}+(1-u)^{\alpha-\beta-1}
\end{gathered}
\]

\section*{References}\label{references}
\addcontentsline{toc}{section}{References}

\phantomsection\label{refs}
\begin{CSLReferences}{1}{0}
\bibitem[\citeproctext]{ref-chalabi2012FlexibleDistributionModeling}
Chalabi, Yohan, David J Scott, and Diethelm Wuertz. 2012. {``Flexible
Distribution Modeling with the Generalized Lambda Distribution.''}
Working Paper MPRA Paper No. 43333,. Zurich, Switzerland: ETH.

\bibitem[\citeproctext]{ref-keelin2016MetalogDistributions}
Keelin, Thomas W. 2016. {``The {Metalog Distributions}.''}
\emph{Decision Analysis} 13 (4): 243--77.
\url{https://doi.org/10.1287/deca.2016.0338}.

\end{CSLReferences}



\end{document}
